% $Id: print-bib.tex 73 2006-03-07 16:41:18Z vilhu001 $
% $URL: http://repository.ciser.cornell.edu/private/proposals/general/print-bib.tex $
% !TeX TXS-program:bibliography = txs:///biber

\documentclass{article}
	\newcommand{\mytitle}{Bibliography \today}
\usepackage{fullpage}
\usepackage[utf8]{inputenc}
\usepackage{etoolbox}
% ======== SET THIS FLAG to indicate whether to use a local or remote bib file
\newtoggle{local}
%\togglefalse{local}
\toggletrue{local}

\usepackage[%
plainpages,%
colorlinks,%
citecolor=blue,%
urlcolor=blue,%
filecolor=black,%
linkcolor=blue,%
pdfpagemode=UseOutlines,%
pdfauthor={Lars Vilhuber},%
pdfsubject={Bibliography},%
]{hyperref}
%\usepackage[citestyle=authoryear,maxnames=10,backend=biber]{biblatex}
\usepackage[style=chicago-authordate,sorting=ydnt,maxnames=10,backend=biber]{biblatex}

%\nocite{*}
\author{Lars Vilhuber }
\date{\today}


\iftoggle{local}{
	\addbibresource{../bib/self.bib}
	\addbibresource{../bib/self.bib}
	\title{Bibliography}
}%else
{
	\addbibresource[location=remote]{https://www.vilhuber.com/lars/wp-content/papercite-data/bib/self.bib}
	%\addbibresource[location=remote]{https://www2.vrdc.cornell.edu/news/wp-content/papercite-data/bib/synlbd-data.bib}
	%\addbibresource[location=remote]{https://www2.vrdc.cornell.edu/news/wp-content/papercite-data/bib/synlbd-documents.bib}
	\title{\mytitle}
}
\nocite{*}

\begin{document}
	\maketitle
Find the latest version of this document at \url{https://www.vilhuber.com/lars/}

\tableofcontents	

%	\begin{quote}
%		\footnotesize
%		This bibliography can be easily recreated by cloning our git repository at \url{https://github.com/ncrncornell/sdsbib}.
%		\iftoggle{local}{This version was created from the BIB file in the repository as of \today. A flag can be set in the preamble of this TEX file to download a remote copy from}{The bibliographic database was downloaded on \today\  directly from} \url{https://www2.vrdc.cornell.edu/news/wp-content/papercite-data/bib/ssb-data.bib} (other locations in the \LaTeX\  source code).
%		\iftoggle{local}{}{A flag can be set in the preamble of this TEX file to use the local checked out version in the git repository.}
%		
%	\end{quote}
	
\defbibfilter{papers}{
	type=article or
	type=inproceedings or
	type=conference or
	type=proceedings or
	type=incollection or 
	type=inbook
}

\defbibfilter{crossref}{
	type=book
}
	
	\printbibliography[filter=papers,title={Articles},heading=bibnumbered]
	\printbibliography[type=report,title={Preprints, Working papers, and Data},heading=bibnumbered]
	\printbibliography[type=online,title={Online},heading=bibnumbered]
	\printbibliography[nottype=report,%
	nottype=article,%
	nottype=inproceedings,%
	nottype=incollection,%
	nottype=conference,%
	nottype=proceedings,%
	nottype=inbook,%
	nottype=book,%
	nottype=online,
	title={Other publications},heading=bibnumbered]
	
	\printbibliography[filter=crossref,title={Referenced Works},heading=bibnumbered]


\end{document}

%%% Local Variables: 
%%% mode: latex
%%% TeX-master: t
%%% End: 
